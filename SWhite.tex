


\title{Analysis of longitudinal data:\\MFQ and brain structure}
%\subtitle{spatio-temporal modelling in neuroimaging}
\author{Kirstie Whitaker, Petra Vertes, Simon R.~White, Ed Bullmore}
%\institute{MRC Biostatistics Unit, University of Cambridge}
\date{2018/Mar/09}

\begin{document}

\maketitle








\begin{frame}{Longitudinal data}
  
  \begin{block}{Benefits}
    \begin{itemize}
    \item Separate within and between subject variability.
    \item Infer temporal effects \emph{(not necessarily causal, also
        the age-period-cohort problem)}.
    \item Obtain subject-specific effects.
    \end{itemize}
  \end{block}

  \pause

  \begin{block}{New problems (beyond cross-sectional data)}
    \begin{itemize}
    \item Incomplete data due to drop out (MCAR/MAR/MNAR).
    \item Need to account for repeated measures within analysis.
    \item Potentially more complex methods.
    \end{itemize}
  \end{block}
  \pause
  \begin{block}{Design}
    NSPN uses an Accelerated Longitudinal Design
  \end{block}

\end{frame}

\begin{frame}{Analysis}

  \begin{block}{(Generalised) Linear mixed-effects modelling}
    \begin{itemize}
    \item Longitudial data fits within random-effect structure.
    \item Can accommodate incomplete data mechanisms.
    \item Well established.
    \item Normal assumption on random-effects.
    \end{itemize}
  \end{block}
  
  \begin{block}{Generalised Estimating Equations}
    \begin{itemize}
    \item Robust to mis-specified random-effects.
    \item Variants allow more esoteric random-effects.
    \item Inference on marginal model.
    \item Does not give subject-level effects.
    \end{itemize}
  \end{block}
  
\end{frame}

\begin{frame}{Non-straight line relationships}
  
  \begin{block}{Fractional Polynomials (Royston \&Altman, 94)}
    A pre-defined set of transformations, specifically:
    \begin{gather*}
      \text{1\textsuperscript{st} order, } x^{p} \quad\text{and}\quad
      \text{2\textsuperscript{nd} order, } x^{p} + x^{q}\\
      p,q \in \{-2,-1,-\frac{1}{2},0,\frac{1}{2},1,2,3\}
    \end{gather*}
  \end{block}
  \pause
  \begin{adjustbox}{center}
    \includegraphics[height=0.5\textheight]{Rplots/FP.pdf}
  \end{adjustbox}%
\end{frame}

\begin{frame}{Model selection}
  \begin{block}{Many possible models}
    \begin{itemize}
    \item 1\textsuperscript{st} order FP models: 8
    \item 2\textsuperscript{nd} order FP models: 36
    \item With and without a Sex-interaction
    \end{itemize}
    \begin{gather*}
      y \sim FP(\text{Age}) + \text{Sex} + \text{Centre} +
      fp(\text{Age})\times\text{Sex} + \text{rand-effects}
    \end{gather*}
  \end{block}%
  

  \pause

  \begin{block}{Statistical criteria}
    \begin{adjustbox}{minipage=0.7\textwidth,center}
      \begin{itemize}
      \item[AIC] Akaike information criterion
      \item[BIC] Bayesian information criterion
      \end{itemize}
    \end{adjustbox}

  \end{block}


  \emph{Aside: Maximum Likelihood for model selection, Restricted
    Maximum Likelihood for inference}
\end{frame}


\begin{frame}{Multi-stage design}
  
  \begin{block}{Questionnaire stage}
    \begin{itemize}
    \item Accelerated Longitudinal Design
    \item Wider sampling to meet second stage criteria ($N=2000$)
    \end{itemize}
  \end{block}

  \begin{block}{Imaging stage}
    \begin{itemize}
    \item Balanced sampling across age and sex
    \item Sub-sample ($N=300$)
    \end{itemize}
  \end{block}


\end{frame}

\begin{frame}{Linking stages}
  \begin{block}{Joint modelling}
    \begin{itemize}
    \item Large and complex model
    \item Substantial
    \end{itemize}    
  \end{block}

  \begin{block}{Multi-stage analysis}
    \begin{itemize}
    \item Analyse each stage separately
    \item Propogate results from one stage to the next
    \end{itemize}    
  \end{block}
  
\end{frame}


\begin{frame}{MFQ}
  
  \begin{block}{``Best'' model (BIC = 46988.39)}
    \begin{gather*}
      \text{MFQ}_{ij} = \underbrace{\beta_0 + \beta_S \text{Sex}_i +
        \beta_A \text{Age}_{ij}^{-2}}_{\text{main-effects}} +
      \underbrace{\beta_{SA}
        \left(\text{Sex}_i\times\text{Age}_{ij}^{-2}\right)}_{\text{Interaction}}
      \\+ \underbrace{u_{i0} + u_{iA}
        \text{Age}_{ij}^{-2}}_{\text{random-effects}}
    \end{gather*}
  \end{block}

  \begin{block}{``Simple'' model}
    \begin{gather*}
      \text{MFQ}_{ij} = \underbrace{\beta_0 + \beta_S \text{Sex}_i +
        \beta_A \text{Age}_{ij}}_{\text{main-effects}} +
      \underbrace{\beta_{SA}
        \left(\text{Sex}_i\times\text{Age}_{ij}\right)}_{\text{Interaction}}
      \\+ \underbrace{u_{i0}}_{\text{random-effects}}
    \end{gather*}
  \end{block}

\end{frame}

\begin{frame}{Equivalent MFQ models ($\delta_{\text{BIC}}<2$)}

  \begin{block}{2\textsuperscript{nd} model (BIC = 46988.69)}
    \begin{gather*}
      \text{MFQ}_{ij} = \underbrace{\beta_0 + \beta_S \text{Sex}_i +
        \beta_A \text{Age}_{ij}^{-1}}_{\text{main-effects}} +
      \underbrace{\beta_{SA}
        \left(\text{Sex}_i\times\text{Age}_{ij}^{-1}\right)}_{\text{Interaction}}
      \\+ \underbrace{u_{i0} + u_{iA}
        \text{Age}_{ij}^{-1}}_{\text{random-effects}}
    \end{gather*}
  \end{block}
  \begin{block}{3\textsuperscript{rd} model (BIC = 46989.73)}
    \begin{gather*}
      \text{MFQ}_{ij} = \underbrace{\beta_0 + \beta_S \text{Sex}_i +
        \beta_A \text{Age}_{ij}^{-\frac{1}{2}}}_{\text{main-effects}} +
      \underbrace{\beta_{SA}
        \left(\text{Sex}_i\times\text{Age}_{ij}^{-\frac{1}{2}}\right)}_{\text{Interaction}}
      \\+ \underbrace{u_{i0} + u_{iA}
        \text{Age}_{ij}^{-\frac{1}{2}}}_{\text{random-effects}}
    \end{gather*}
  \end{block}

\end{frame}

\begin{frame}{MFQ plots}
  \only<1>{\begin{adjustbox}{center}
      \includegraphics[width=\linewidth]{Rplots/Tident.MFQ3-full.pdf}
    \end{adjustbox}\par}
  \only<2>{\begin{adjustbox}{center}
      \includegraphics[width=\linewidth]{Rplots/Tident.MFQ3-narrow.pdf}
    \end{adjustbox}\par}
  \only<3>{\begin{adjustbox}{center}
      \includegraphics[width=\linewidth]{Rplots/Tident.MFQ3-raw.pdf}
    \end{adjustbox}\par}
  \only<4>{\begin{adjustbox}{center}
      \includegraphics[width=\linewidth]{{Rplots/Tident.MFQ3-raw+best.pdf}}
    \end{adjustbox}\par}
  
\end{frame}

\begin{frame}{Derived measures}
  \begin{adjustbox}{center}
    \includegraphics[width=\linewidth]{{Rplots/Tident.MFQ3-derived.pdf}}
  \end{adjustbox}\par
\end{frame}

\begin{frame}{Brain regions}
  
  \begin{itemize}
  \item Repeat same procedure on every brain region
  \item Using APARC-500 parcellation
  \end{itemize}

  \begin{itemize}
  \item Include \alert{Centre} in model
  \item Only include random-intercept
  \end{itemize}

\end{frame}

\begin{frame}{Brain structures}


  \begin{adjustbox}{center}
    \includegraphics[width=0.8\textwidth]{../../pysurfer/500aparc_MT_frac+030_mean/DATraw-m14a_SETall_MODraw/UisVis/SiC0A+10xxxRxxx_SiCmA+10xxxRxxx/PNGS/Tident_MFQ_mid-point_corr_ranef-1_m_pial_classic_FourHorBrains.png}
  \end{adjustbox}
  \begin{adjustbox}{center}
    \footnotesize{\usebeamercolor[fg]{item}Fig:} \emph{MFQ(mid-point)
      association with MT(rand-intercept)}
  \end{adjustbox}

  \begin{adjustbox}{center}
    \includegraphics[width=0.8\textwidth]{../../pysurfer/500aparc_thickness/DATraw-m14a_SETall_MODraw/UisVis/SiC0A+10xxxRxxx_SiCmA+10xxxRxxx/PNGS/Tident_MFQ_mid-point_corr_ranef-1_m_pial_classic_FourHorBrains.png}
  \end{adjustbox}
  \begin{adjustbox}{center}
    \footnotesize{\usebeamercolor[fg]{item}Fig:} \emph{MFQ(mid-point)
      association with CT(rand-intercept)}
  \end{adjustbox}

\end{frame}


\begin{frame}{Next steps}
  
  \begin{block}{``best'' brain model}
    \begin{itemize}
    \item Currently selecting global ``best'' model across all brain regions
    \item Potentially averaging, which may attenuate signal
    \end{itemize}
  \end{block}

  \begin{block}{Linking MFQ to brain changes}
    \begin{itemize}
    \item Cluster brain regions based on model fits
    \item Cluster based on `similar' trajectories
    \end{itemize}
  \end{block}

  
\end{frame}




\begin{frame}{Acknowledgments}

  \begin{adjustbox}{center}
    \begin{minipage}{0.24\linewidth}
      \includegraphics[width=0.45\columnwidth,height=0.45\columnwidth,keepaspectratio]{thanks/{kirstie.whitaker}}

      \begin{minipage}[b]{\linewidth}
        \scriptsize{}Kirstie Whitaker
      \end{minipage}
    \end{minipage}
    \begin{minipage}{0.24\linewidth}
      \includegraphics[width=0.45\columnwidth,height=0.45\columnwidth,keepaspectratio]{thanks/petra.vertes}

      \begin{minipage}[b]{\linewidth}
        \scriptsize{}Petra Vertes
      \end{minipage}
    \end{minipage}
    \begin{minipage}{0.24\linewidth}
      \includegraphics[width=0.45\columnwidth,height=0.45\columnwidth,keepaspectratio]{thanks/{ed.bullmore}}

      \begin{minipage}[b]{\linewidth}
        \scriptsize{}Ed Bullmore
      \end{minipage}
    \end{minipage}
  \end{adjustbox}

  \vspace{2em}

  \begin{adjustbox}{center}
    \begin{tikzpicture}[node distance=1mm]
      \node (NSPN) {\includegraphics[height=2em]{logo/NSPN_logo}};
      \node (BSU) [left=of NSPN]
      {\includegraphics[height=2em]{logo/MRC_Biostatistics_Cambridge}};
      \node (ATI) [right=of NSPN]
      {\includegraphics[height=2em]{logo/ati}};
    \end{tikzpicture}
  \end{adjustbox}

\end{frame}







\end{document}

%%% Local Variables: 
%%% mode: latex
%%% TeX-master: "SWhite-slides"
%%% End: 
